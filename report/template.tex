\input{preamble.tex}

\title{CI/CD with Jenkins and Beaker}
\author{Thomas Løkkeborg} % TODO add studnr

\begin{document}

\maketitle

\abstract{TBA}

\thispagestyle{empty}

\clearpage
\pagenumbering{roman}
\setcounter{page}{1}
\tableofcontents

\clearpage
\pagenumbering{arabic}

\section{Introduction}

TBA

%This is a short template you can use for your project report. Feel free to
%make your own modifications to the structure. 
%
%Join together to form project groups of max three people in each group (you can
%also be just one if you prefer to work alone).
%
%For the topic you choose write a report of ca 5-15 pages plus appendices (and prepare a 5-10
%minute presentation).
%
%Try to write as scientifically correct as you can. Write objectively, do not
%write like you are telling a story: "first we did this, then we did that, ...".
%Use correct citations, here are some examples of how to cite correctly:
%\begin{itemize}
%\item Dag Langmyrs \LaTeX book~\cite{Langmyr:03}
%\item Journal articles like~\cite{Klein:09}
%\item Conference proceedings articles like~\cite{Begnum:07} 
%\item Wikipedia pages like~\cite{wikipedia:kerberos:10}
%\end{itemize}
%
%NOTE: it is EXTREMELY important that you write in your own words, and not
%just translate something you find. 
%
%If you want to write ``the perfect introduction'' I strongly suggest you
%read Claerbout's ``Scrutiny of the introduction''~\cite{Claerbout:95}.
%
%Describe what you will do in the introduction chapter (What are the goals of
%you project?).

\section{Background technology}

TBA

%Describe the technology involved, you do not have to explain something you have
%learned in the course, but explain any additional technology that you use in
%the prpoject.

\section{Survey of similar projects}

TBA

%There are always someone who has done what you are going to do, or
%something related. Breifly describe at least one or two similar projects,
%to motivate what your contribution will be (Google/DuckDuckGo the best you can
%and find similar projects).

\section{Description of your work}

TBA

%Describe what you have done (or are doing), relate it the the previous work
%you described in the previous chapter if appropriate.

\section{Results and discussion}

TBA

%This is the longest chapter, feel free to include some figures. Include
%discussion of all problems you have encountered.

\section{Security aspects}

TBA

%Provide a brief security analysis. Are you opening any new attack vectors? What
%are the risks? Are there sensitive data? Are clients and servers mutually
%authenticating each other? etc etc

\section{Conclusions}

TBA

%So what is the punchline, does it work or not? Do you recommend doing this
%for others?

\bibliographystyle{acmdoi}
\bibliography{template}

\clearpage
\appendix

\section{Code}

TBA

%\begin{verbatim}
%# this is just the verbatim environment, really should use
%# a proper code environement
%
%class profile::base_linux {
%
%  $linux_sw_pkg = hiera('base_linux::linux_sw_pkg')
%
%# careful when configuring ntp to avoid misuse (opening for DDOS)
%
%  class { 'ntp':
%    servers   => [ 'ntp.hig.no' ],
%    restrict  => [
%      'default kod nomodify notrap nopeer noquery',
%      '-6 default kod nomodify notrap nopeer noquery',
%    ],
%  }
%  class { 'timezone':
%    timezone => 'Europe/Oslo',
%  }
%
%  package { $linux_sw_pkg:
%    ensure => latest,
%  }
%
%}
%\end{verbatim}
%
\end{document}
